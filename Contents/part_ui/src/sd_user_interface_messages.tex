\begin{SCn}
\scnsectionheader{Предметная область и онтология сообщений, входящих в ostis-систему и выходящих из неё}
\begin{scnsubstruct}
\scnrelfrom{соавтор}{Садовский М.Е.}

\scnheader{сообщение}
\begin{scnrelfromlist}{разбиение}
	\scnsuperset{сообщение пользователя системы}
	\begin{scnindent}
		\scnsuperset{сообщение пользователя ostis-системы}
	\end{scnindent}
	\scnsuperset{сообщение системы}
\end{scnrelfromlist}
\begin{scnrelfromlist}{разбиение}
	\scnsuperset{атомарное сообщение}
	\scnsuperset{неатомарное сообщение}
\end{scnrelfromlist}
\begin{scnrelfromlist}{разбиение}
	\scnsuperset{сообщение на естественном языке}
	\scnsuperset{сообщение на искусственном языке}
\end{scnrelfromlist}
\scnsuperset{графическое сообщение}
\begin{scnindent}
	\scnidtf{сообщение, содержащее графическую информацию}
	\scnsuperset{видео-сообщение}
	\begin{scnindent}
		\scnidtf{сообщение, содержащее видео-информацию}
	\end{scnindent}
\end{scnindent}
\scnsuperset{аудио-сообщение}
\begin{scnindent}
	\scnidtf{сообщение, представленное в звуковом формате}
\end{scnindent}
\scnsuperset{обонятельное сообщение}
\begin{scnindent}
	\scnidtf{сообщение, содержащее информацию о запахах}
\end{scnindent}
\scnsuperset{текстовое сообщение}
\begin{scnindent}
	\scnidtf{сообщение, содержащее текстовую информацию}
\end{scnindent}
\scnsuperset{сообщение, требующее трансляции}
\begin{scnindent}
	\scnidtf{сообщение, которое необходимо сформировать системой для дальнейшей передачи его пользователю}
\end{scnindent}
\scnsuperset{протранслированное сообщение}
\begin{scnindent}
	\scnidtf{сообщение, которое было сформировано системой для дальнейшей передачи его пользователю}
\end{scnindent}
\scnheader{решатель задач пользовательского интерфейса ostis-систем}
\scnsuperset{sc-агенты}
\begin{scnindent}
	\scnidtf{обеспечивают работу пользователя с компонентами пользовательского интерфейса ostis-системы}
\end{scnindent}
\scnsuperset{семантическая составляющая}
\begin{scnindent}
	\scntext{задача}{определение того, знаком какой сущности является отображаемый на экране компонент}
\end{scnindent}
\scnsuperset{прагматическая составляющая}
\begin{scnindent}
	\scnidtf{рассматривает прикладной аспект (аспект применения) отображаемого на экране компонента}
\end{scnindent}
\scntext{примечание}{На уровне sc-памяти имеет значение только семантическая составляющая, однако данный факт не влияет на процесс эксплуатации системы пользователем, поскольку обе составляющие отражают разные стороны одного и того же знака некоторой сущности.}
\scnheader{внутреннее действие системы}
\scnsuperset{внутреннее действие ostis-системы}
\scnheader{внутреннее действие ostis-системы}
\scnidtf{действие в sc-памяти}
\scnidtf{действие, выполняемое в sc-памяти}
\scnheader{действие в sc-памяти}
\scnsuperset{действие в sc-памяти, инициируемое вопросом}
\scnsuperset{действие редактирования базы знаний ostis-системы}
\scnsuperset{действие установки режима ostis-системы}
\scnsuperset{действие редактирования файла, хранимого в sc-памяти}
\scnsuperset{ действие интерпретации программы, хранимой в sc памяти}
\scnheader{база знаний пользовательского интерфейса ostis-системы}
\begin{scnrelfromset}{задачи}
	\scnitem{обработка пользовательских действий}
	\scnitem{интерпретация модели базы знаний пользовательского интерфейса ostis-системы(построение пользовательского интерфейса)}
\end{scnrelfromset}

\bigskip
\end{scnsubstruct}
\scnendcurrentsectioncomment
\end{SCn}